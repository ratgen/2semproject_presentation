\documentclass{article}

\usepackage[utf8]{inputenc}
\usepackage[danish]{babel}
\usepackage{float}
\usepackage{fancyhdr}
\usepackage{amsmath}
\usepackage{color}
\usepackage{listings}
\usepackage{graphicx}
\usepackage{pdfpages}
\usepackage{booktabs}
\usepackage{listingsutf8}

\title{Semesterprojekt \\ Krediteringssystem til TV 2}
\author{Peter Heilbo Ratgen}
\date{\today}

\hfuzz=10pt

\begin{document}
\maketitle

\section{Notater til præsentation}%
\label{sec:notater_til_praesentation}

\paragraph{Hvad fandt du særligt interressant?}%
\label{par:hvad_fandt_du_saerligt_interressant}
Særligt interressant fandt jeg processen i og omkring lagdelingen af systemet.
Vigtigheden af designprocessen understreges af den tid vi brugte til at
refaktorere system således at der var en egentlig lagdeling. At lave om i
systemet var svært i begyndelsen før refaktoringen, siden vi havde lavet en
monolitisk applikation. Vores system havde en tæt kobling i mellem
præsentationslaget og domænelaget. 

Denne tætte kobling medførte at det var sværere at lave ændringer i systemet,
uden at skulle lave ændringer mange steder i koden.

Ignnem lagdelingen lavede vi konkrete vurderinger af hvad de hørte til hvor. Fx
er var det meget tydeligt at forrentningsdata ikke hører til i og omkring
præsentationlaget. Vi lavede også en separering af data fra de egentlige
dataobjekter, således vi kun afhang af interfaces.

Denne fornyede designprocess valgte vi også at anvende
model-view-controller designmønstreret til at understøtte lagdelingen, samt at
kunne separere views og controllers.

\end{document}
